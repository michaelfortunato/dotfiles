\documentclass[10pt, letterpaper]{article}
\usepackage{amsmath}
\usepackage{amssymb}
\usepackage{amsthm}
\usepackage{mathtools}
%
\newtheorem{theorem}{Theorem}[section]
\newtheorem{lemma}[theorem]{Lemma}
\theoremstyle{definition}
\newtheorem{definition}{Definition}[section]
\theoremstyle{remark}
\newtheorem*{remark}{Remark}
%
\begin{document}
% MNF Default Math Latex Document
\title{Notes from CMSC-353}
\author{Michael Newman Fortunato}
% FIXME: \institute{} is broken
\maketitle

\subsection{SVD and Ridge Regression}
We now decompose ridge regression into its essence, its SVD.
Perhaps there are other ways to view $\mathbf{X}$ that gives us insight,
but SVD is an essential viewpoint.

\begin{remark}
   Recall that Singular Value Decomposition of a real matrix $\mathbf{X}$
   can be viewed as applying the $V^{T}$ rotation, doing a sheering, 
   scaling along the rows of $\mathbf{V}^{T}$ by potentially different 
   amounts, and then doing a final rotation in $\mathbf{U}$.
   \[ X = U \Sigma V^{T} \]
\end{remark}

\subsubsection{Viewing Ridge Regression through SVD reveals its strength}

We can see in more ways than one, why ridge regression is advantages in certain
cases, by expressing $X$ by its SVD in the gradient equation of the ridge loss.

\begin{equation}
    \hat w = \left(X^{T}X + \lambda I \right)^{-1}X^{T}y
\end{equation}

\paragraph{Questions}
Some questions come to mind 
\begin{itemize}
  \item How do we know $\left(X^{T}X - \lambda I\right)^{-1}$ is invertible?
\end{itemize}

\newcommand{\svd}[0]{X \Sigma V^T}

\begin{align*}
    \left(X^{T}X - \lambda I\right)^{-1}X^{T}y  \\
    \left(\left(\svd\right)^{T}X - \lambda I\right)^{-1}X^{T}y 
\end{align*}


  
\end{document}
