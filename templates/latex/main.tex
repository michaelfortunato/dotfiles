\documentclass{article}

% if you need to pass options to natbib, use, e.g.:
%     \PassOptionsToPackage{numbers, compress}{natbib}
% before loading neurips_2022

% ready for submission
\usepackage[preprint]{neurips_2024}

% to compile a preprint version, e.g., for submission to arXiv, add add the
% [preprint] option:
%     \usepackage[preprint]{neurips_2022}

% to compile a camera-ready version, add the [final] option, e.g.:
%     \usepackage[final]{neurips_2022}

% to avoid loading the natbib package, add option nonatbib:
%    \usepackage[nonatbib]{neurips_2022}

\usepackage[utf8]{inputenc} % allow utf-8 input
\usepackage[T1]{fontenc}    % use 8-bit T1 fonts
\usepackage{hyperref}       % hyperlinks
\usepackage{url}            % simple URL typesetting
\usepackage{booktabs}       % professional-quality tables
\usepackage{amsfonts}       % blackboard math symbols
\usepackage{nicefrac}       % compact symbols for 1/2, etc.
\usepackage{microtype}      % microtypography
\usepackage{xcolor}         % colors
\newcommand{\new}[1]{{\color{red} #1}}

\title{SymD: Discovering Discrete Local Symmetries In Objects}

% The \author macro works with any number of authors. There are two commands
% used to separate the names and addresses of multiple authors: \And and \AND.
%
% Using \And between authors leaves it to LaTeX to determine where to break the
% lines. Using \AND forces a line break at that point. So, if LaTeX puts 3 of 4
% authors names on the first line, and the last on the second line, try using
% \AND instead of \And before the third author name.

\author{%
  Michael Newman Fortunato \thanks{Personal webpage:
    \texttt{mnf.ai}. Personal email:
    \texttt{michael.n.fortunato@gmail.com}.
  } \\
  Department of Computer Science\\
  University of Chicago\
  Chicago, IL 60637\\
  \texttt{michaelfortunato@uchicago.edu} \\
  % examples of more authors
  % \And
  % Coauthor \\
  % Affiliation \\
  % Address \\
  % \texttt{email} \\
  % \AND
  % Coauthor \\
  % Affiliation \\
  % Address \\
  % \texttt{email} \\
  % \And
  % Coauthor \\
  % Affiliation \\
  % Address \\
  % \texttt{email} \\
  % \And
  % Coauthor \\
  % Affiliation \\
  % Address \\
  % \texttt{email} \\
}

\begin{document}

\maketitle

\begin{abstract}
  A symmetry of an object is often localized to just a region on that object.
  In other words, and this is especially true for objects in nature,
  objects are usually not entirely symmetrical. Instead the symmetries occurn
  \textit{within} the object. We call these symmetries,
  \textit{local symmetries}. Because local symmetries are the most prevalent
  in nature, we wish to, given an object, automatically discover
  local symmetries. Using the group theory, we formalize the notion of local
  symmetry, develop a methodology for discovering discete local symmetries in
  objects, and present results showing the effectivness of our system.
  The code for this paper is publicly available at
  \texttt{github.com/michaelfortunato/SymD}.
  The experiments are entirely reproducible, and setups are given for multiple
  platforms. View the project's README at
  \texttt{github.com/michaelfortunato/SymD/blob/main/README.typ}.
\end{abstract}
\noindent Please create a copy of this template for your write-up.
Follow the format instructions in Section \ref{gen_inst} to write
your report. A typical machine learning paper is 8 pages (not
including references). You can find more resources on the
\href{https://docs.google.com/document/d/16R1E2ExKUCP5SlXWHr-KzbVDx9DBUclra-EbU8IB-iE/edit#heading=h.16t67gkeu9dx}{How
to ML Papers}.

\textbf{Hints on writing}:
\begin{itemize}
  \item Each paragraph should convey one main message, typically in
    the beginning sentence of the paragraph (topical sentence).
  \item  Use simple sentences with fewer compounds or clauses. The
    simpler, the better. Avoid passive tense and use active tense.
  \item Use PDF for your figures to ensure the resolution of your images.
  \item The captions of Figures and Tables should contain sufficient
    details to be self-explanatory (readers can understand the
    figure without referring to the text).
\end{itemize}
Please read the instructions below carefully and follow them
faithfully. You can keep the section headers but delete the
instructional texts.

\section{Introduction}
Introduce the problem that you are trying to solve and write 1-2
paragraphs for each of the sub-sections. Provide a high-level
summary in this section instead of detailed descriptions.

\paragraph{Problem Definition.}
Describe  what is the problem you want to solve in concise language.
You may use figures to help your explanation.

\paragraph{Problem Significance.}
Explain  why is your problem relevant to generative AI. How is your
project useful to deep learning and our society if you can solve it.

\paragraph{Technical Challenge.}
Why is this project that you are proposing difficult? Think about
challenges from (1) data gathering (2) model design (3)
implementation (4) engineering (5) evaluation.

\paragraph{State-of-the-Art.}
Find and cite (use
  \href{https://www.overleaf.com/learn/latex/Bibliography_management_with_bibtex}{BibTex
}) state-of-the-art works or papers that have tried to solve a
similar problem (use \href{https://scholar.google.com/}{Google
Scholar} to search). Point out the limitations of the state-of-the-art.

\paragraph{Contributions.}
Summarize your contributions to this problem with bullet points.
\begin{itemize}
  \item Contribution 1: solve a novel problem
  \item Contribution 2: improve the generative AI performance
  \item $\cdots$
\end{itemize}

\section{Related Work}
Provide a detailed discussion of the state-of-the-art using the
references that you have found. The description can overlap with the
state-of-the-art sub-section in the Introduction but should contain
more technical details. Pick 2 or 3 topics that are adjacent to your
problem. For example, if you are trying build a better deep
generative model for speech synthesis, then the related topics can
include (1) deep generative models and (2) speech synthesis.

\paragraph{Related Topic 1 [Replace with the topic keywords].} Write
a few sentences summarizing the key ideas of each of the related
works from State-of-the-Art in the previous section. Highlight the
technical differences between the related work and your proposed method.

\textbf{Related Topic 2 [Replace with the topic keywords].}

\section{Methodology}

\paragraph{Problem Setting.}
Formally introduces the problem setting and notation (Formalism) for
your method. Highlights any specific assumptions that are made that
are unusual. You are encouraged to use figures or equations to
better explain the input and output.

\paragraph{Idea Summary.}
Summarize your idea to solve the problem.  How is it different from
the state-of-the-art method?  Is there any risk associated with your
idea? Provide alternative solutions if your idea does not work out.

\paragraph{Description.}
Explain in the mathematical language of your methodology. Use the
(probabilistic) notations introduced in the problem definition, and
explain what distributions your generative AI model is learning
(Hint: a distribution over high-dimensional data). Which generative
model do you plan to use to learn this distribution?    Draw an
overview figure to illustrate the details of your idea.

\paragraph{Implementation.}
Provide technical details for your implementation. What deep
learning framework did you use (Keras, Tensorflow or Pytorch)? Did
you implement your model based on an existing codebase? If yes, pls
include the reference and highlight your contribution to the
codebase. What model architectures have you tried? What engineering
tricks (e.g. BatchNorm, Drop-Out, Early Stopping) did you implement?
How did you tune the hyper-parameters?

\section{Experiments}
\paragraph{Datasets and Tools.}
List the datasets or tools  you used for this project. Only describe
the datasets that you have used in the experiments. Provide details
in numbers about the size, and dimension of the datasets. Provide
links to these resources if available.

\paragraph{Baselines.}
List the  baseline methods  that you plan to include for comparison.
Explain your rationale for choosing these baselines. They can be
publicly available solutions from existing papers or even
open-source projects on Github. Provide references for these baselines.
{

  \begin{itemize}
    \item Baseline 1 [reference]: 1-2 sentences summary of the baseline.
    \item Baseline 2 [reference]:1-2 sentences summary of the baseline.
    \item $\dots$
  \end{itemize}
}

\paragraph{Evaluation Metrics.}
What are the evaluation metrics that will be used to quantify the
performance? Can you think of quantitative metrics to compare your
method with the baselines?

\paragraph{Quantitative Results.}
Report quantitative results that you have achieved with the baselines.
Report quantitative results of your final implementation of the
proposed idea. Highlight the method that has the best performance
compared to baselines in boldface.

You are encouraged to use figures or tables to better organize your
results. If you don't have any baselines to compare with, you can
also include the simplest solution you can think of to start with.
Examples of quantitative results can include sample quality scores,
model likelihood, run-time comparison, downstream
supervised-learning prediction performance, etc. Remember to include
standard deviation over multiple random runs.

\paragraph{Qualitative Results.}
Report qualitative results that you have achieved with the baselines.
Report qualitative results of your final implementation of the
proposed idea. Explain the qualitative differences between your
method and the baselines.
You are encouraged to use figures or tables to better organize your
results. If you don't have any baselines to compare with, you can
also include the simplest solution you can think of to start with..
Example qualitative results can include visualization of the
generated samples, an interactive demo page, visualization of the
learned model, etc.

% \paragraph{Ablative Studies.}
% Remove certain component of your  model  in order to gain a better
% understanding of your method.  Each time you can  remove one of
% the modules and check the performance of the new model to
% investigate the influence of the removed module:

\section{Conclusion and Discussion}
Provide a few sentences to summarize your project.
\paragraph{Achievements.}
What problem did you solve? What is the significance of your
contribution? Did you contribute to dataset curation, algorithm
improvement or model development?  Summarize your achievement in
quantitative terms. We value the effort that you have devoted to this project.

\paragraph{Lessons Learned.}
What did you learn from this project?  If given more time, what
aspects and how would you improve your project development?
Brainstorm in different aspects including selecting problems,
literature survey, team formulation and technical implementation.
What advice would you give to future students who will work on this
course project?

% \paragraph{Updated Bottlenecks.}
% What are your current bottlenecks of the project? How do you plan
% to resolve them?
%Any changes that you would like to highlight from the proposal?

% \paragraph{Updated Timelines}
% Update the timeline with the milestones to achieve your expected outcome.
% \begin{itemize}
%     \item Week 2: Fill in your milestones and deliverable
%     \item Week 3:
%     \item ...
%     \item Week 10:
% \end{itemize}

\newpage
\section{General formatting instructions}
\label{gen_inst}

The text must be confined within a rectangle 5.5~inches (33~picas) wide and
9~inches (54~picas) long. The left margin is 1.5~inch (9~picas).  Use 10~point
type with a vertical spacing (leading) of 11~points.  Times New Roman is the
preferred typeface throughout, and will be selected for you by default.
Paragraphs are separated by \nicefrac{1}{2}~line space (5.5 points), with no
indentation.

The paper title should be 17~point, initial caps/lower case, bold, centered
between two horizontal rules. The top rule should be 4~points thick and the
bottom rule should be 1~point thick. Allow \nicefrac{1}{4}~inch space above and
below the title to rules. All pages should start at 1~inch (6~picas) from the
top of the page.

For the final version, authors' names are set in boldface, and each name is
centered above the corresponding address. The lead author's name is
to be listed
first (left-most), and the co-authors' names (if different address) are set to
follow. If there is only one co-author, list both author and co-author side by
side.

Please pay special attention to the instructions in Section \ref{others}
regarding figures, tables, acknowledgments, and references.

\section{Headings: first level}
\label{headings}

All headings should be lower case (except for first word and proper nouns),
flush left, and bold.

First-level headings should be in 12-point type.

\subsection{Headings: second level}

Second-level headings should be in 10-point type.

\subsubsection{Headings: third level}

Third-level headings should be in 10-point type.

\paragraph{Paragraphs}

There is also a \verb+\paragraph+ command available, which sets the heading in
bold, flush left, and inline with the text, with the heading followed by 1\,em
of space.

\section{Citations, figures, tables, references}
\label{others}

These instructions apply to everyone.

\subsection{Citations within the text}

The \verb+natbib+ package will be loaded for you by default.  Citations may be
author/year or numeric, as long as you maintain internal
consistency.  As to the
format of the references themselves, any style is acceptable as long as it is
used consistently.

The documentation for \verb+natbib+ may be found at
\begin{center}
  \url{http://mirrors.ctan.org/macros/latex/contrib/natbib/natnotes.pdf}
\end{center}
Of note is the command \verb+\citet+, which produces citations appropriate for
use in inline text.  For example,
\begin{verbatim}
   \citet{hasselmo} investigated\dots
\end{verbatim}
produces
\begin{quote}
  Hasselmo, et al.\ (1995) investigated\dots
\end{quote}

If you wish to load the \verb+natbib+ package with options, you may add the
following before loading the \verb+neurips_2022+ package:
\begin{verbatim}
   \PassOptionsToPackage{options}{natbib}
\end{verbatim}

If \verb+natbib+ clashes with another package you load, you can add
the optional
argument \verb+nonatbib+ when loading the style file:
\begin{verbatim}
   \usepackage[nonatbib]{neurips_2022}
\end{verbatim}

As submission is double blind, refer to your own published work in the third
person. That is, use ``In the previous work of Jones et al.\ [4],''
not ``In our
previous work [4].'' If you cite your other papers that are not
widely available
(e.g., a journal paper under review), use anonymous author names in the
citation, e.g., an author of the form ``A.\ Anonymous.''

\subsection{Footnotes}

Footnotes should be used sparingly.  If you do require a footnote, indicate
footnotes with a number\footnote{Sample of the first footnote.} in the
text. Place the footnotes at the bottom of the page on which they appear.
Precede the footnote with a horizontal rule of 2~inches (12~picas).

Note that footnotes are properly typeset \emph{after} punctuation
marks.\footnote{As in this example.}

\subsection{Figures}

\begin{figure}
  \centering
  \fbox{\rule[-.5cm]{0cm}{4cm} \rule[-.5cm]{4cm}{0cm}}
  \caption{Sample figure caption.}
\end{figure}

All artwork must be neat, clean, and legible. Lines should be dark enough for
purposes of reproduction. The figure number and caption always appear after the
figure. Place one line space before the figure caption and one line space after
the figure. The figure caption should be lower case (except for first word and
proper nouns); figures are numbered consecutively.

You may use color figures.  However, it is best for the figure captions and the
paper body to be legible if the paper is printed in either black/white or in
color.

\subsection{Tables}

All tables must be centered, neat, clean and legible.  The table number and
title always appear before the table.  See Table~\ref{sample-table}.

Place one line space before the table title, one line space after the
table title, and one line space after the table. The table title must
be lower case (except for first word and proper nouns); tables are
numbered consecutively.

Note that publication-quality tables \emph{do not contain vertical rules.} We
strongly suggest the use of the \verb+booktabs+ package, which allows for
typesetting high-quality, professional tables:
\begin{center}
  \url{https://www.ctan.org/pkg/booktabs}
\end{center}
This package was used to typeset Table~\ref{sample-table}.

\begin{table}
  \caption{Sample table title}
  \label{sample-table}
  \centering
  \begin{tabular}{lll}
    \toprule
    \multicolumn{2}{c}{Part}                   \\
    \cmidrule(r){1-2}
    Name     & Description     & Size ($\mu$m) \\
    \midrule
    Dendrite & Input terminal  & $\sim$100     \\
    Axon     & Output terminal & $\sim$10      \\
    Soma     & Cell body       & up to $10^6$  \\
    \bottomrule
  \end{tabular}
\end{table}

\section{Final instructions}

Do not change any aspects of the formatting parameters in the style files.  In
particular, do not modify the width or length of the rectangle the text should
fit into, and do not change font sizes (except perhaps in the
\textbf{References} section; see below). Please note that pages should be
numbered.

\section{Preparing PDF files}

Please prepare submission files with paper size ``US Letter,'' and not, for
example, ``A4.''

Fonts were the main cause of problems in the past years. Your PDF
file must only
contain Type 1 or Embedded TrueType fonts. Here are a few instructions to
achieve this.

\begin{itemize}

  \item You should directly generate PDF files using \verb+pdflatex+.

  \item You can check which fonts a PDF files uses.  In Acrobat
    Reader, select the
    menu Files$>$Document Properties$>$Fonts and select Show All Fonts. You can
    also use the program \verb+pdffonts+ which comes with \verb+xpdf+ and is
    available out-of-the-box on most Linux machines.

  \item The IEEE has recommendations for generating PDF files whose
    fonts are also
    acceptable for NeurIPS. Please see
    \url{http://www.emfield.org/icuwb2010/downloads/IEEE-PDF-SpecV32.pdf}

  \item \verb+xfig+ "patterned" shapes are implemented with bitmap fonts.  Use
    "solid" shapes instead.

  \item The \verb+\bbold+ package almost always uses bitmap fonts.
    You should use
    the equivalent AMS Fonts:
\begin{verbatim}
   \usepackage{amsfonts}
\end{verbatim}
    followed by, e.g., \verb+\mathbb{R}+, \verb+\mathbb{N}+, or
    \verb+\mathbb{C}+
    for $\mathbb{R}$, $\mathbb{N}$ or $\mathbb{C}$.  You can also
    use the following
    workaround for reals, natural and complex:
\begin{verbatim}
   \newcommand{\RR}{I\!\!R} %real numbers
   \newcommand{\Nat}{I\!\!N} %natural numbers
   \newcommand{\CC}{I\!\!\!\!C} %complex numbers
\end{verbatim}
    Note that \verb+amsfonts+ is automatically loaded by the
    \verb+amssymb+ package.

\end{itemize}

If your file contains type 3 fonts or non embedded TrueType fonts, we will ask
you to fix it.

\subsection{Margins in \LaTeX{}}

Most of the margin problems come from figures positioned by hand using
\verb+\special+ or other commands. We suggest using the command
\verb+\includegraphics+ from the \verb+graphicx+ package. Always specify the
figure width as a multiple of the line width as in the example below:
\begin{verbatim}
   \usepackage[pdftex]{graphicx} ...
   \includegraphics[width=0.8\linewidth]{myfile.pdf}
\end{verbatim}
See Section 4.4 in the graphics bundle documentation
(\url{http://mirrors.ctan.org/macros/latex/required/graphics/grfguide.pdf})

A number of width problems arise when \LaTeX{} cannot properly hyphenate a
line. Please give LaTeX hyphenation hints using the \verb+\-+ command when
necessary.

\begin{ack}
  Use unnumbered first level headings for the acknowledgments. All
  acknowledgments
  go at the end of the paper before the list of references.
  Moreover, you are required to declare
  funding (financial activities supporting the submitted work) and
  competing interests (related financial activities outside the
  submitted work).
  More information about this disclosure can be found at:
  \url{https://neurips.cc/Conferences/2022/PaperInformation/FundingDisclosure}.

  Do {\bf not} include this section in the anonymized submission,
  only in the final paper. You can use the \texttt{ack} environment
  provided in the style file to autmoatically hide this section in
  the anonymized submission.
\end{ack}

\section*{References}

References follow the acknowledgments. Use unnumbered first-level heading for
the references. Any choice of citation style is acceptable as long as you are
consistent. It is permissible to reduce the font size to \verb+small+ (9 point)
when listing the references.
Note that the Reference section does not count towards the page limit.
\medskip

{
  \small

  [1] Alexander, J.A.\ \& Mozer, M.C.\ (1995) Template-based algorithms for
  connectionist rule extraction. In G.\ Tesauro, D.S.\ Touretzky and T.K.\ Leen
  (eds.), {\it Advances in Neural Information Processing Systems 7},
  pp.\ 609--616. Cambridge, MA: MIT Press.

  [2] Bower, J.M.\ \& Beeman, D.\ (1995) {\it The Book of GENESIS: Exploring
    Realistic Neural Models with the GEneral NEural SImulation
  System.}  New York:
  TELOS/Springer--Verlag.

  [3] Hasselmo, M.E., Schnell, E.\ \& Barkai, E.\ (1995) Dynamics of
  learning and
  recall at excitatory recurrent synapses and cholinergic modulation in rat
  hippocampal region CA3. {\it Journal of Neuroscience} {\bf 15}(7):5249-5262.
}

%%%%%%%%%%%%%%%%%%%%%%%%%%%%%%%%%%%%%%%%%%%%%%%%%%%%%%%%%%%%
\section*{Checklist}

%%% BEGIN INSTRUCTIONS %%%
The checklist follows the references.  Please
read the checklist guidelines carefully for information on how to answer these
questions.  For each question, change the default \answerTODO{} to
\answerYes{},
\answerNo{}, or \answerNA{}.  You are strongly encouraged to include a {\bf
justification to your answer}, either by referencing the appropriate section of
your paper or providing a brief inline description.  For example:
\begin{itemize}
  \item Did you include the license to the code and datasets?
    \answerYes{See Section~\ref{gen_inst}.}
  \item Did you include the license to the code and datasets?
    \answerNo{The code and the data are proprietary.}
  \item Did you include the license to the code and datasets? \answerNA{}
\end{itemize}
Please do not modify the questions and only use the provided macros for your
answers.  Note that the Checklist section does not count towards the page
limit.  In your paper, please delete this instructions block and only keep the
Checklist section heading above along with the questions/answers below.
%%% END INSTRUCTIONS %%%

\begin{enumerate}

  \item For all authors...
    \begin{enumerate}
      \item Do the main claims made in the abstract and introduction
        accurately reflect the paper's contributions and scope?
        \answerTODO{}
      \item Did you describe the limitations of your work?
        \answerTODO{}
      \item Did you discuss any potential negative societal impacts
        of your work?
        \answerTODO{}
      \item Have you read the ethics review guidelines and ensured
        that your paper conforms to them?
        \answerTODO{}
    \end{enumerate}

  \item If you are including theoretical results...
    \begin{enumerate}
      \item Did you state the full set of assumptions of all
        theoretical results?
        \answerTODO{}
      \item Did you include complete proofs of all theoretical results?
        \answerTODO{}
    \end{enumerate}

  \item If you ran experiments...
    \begin{enumerate}
      \item Did you include the code, data, and instructions needed
        to reproduce the main experimental results (either in the
        supplemental material or as a URL)?
        \answerTODO{}
      \item Did you specify all the training details (e.g., data
        splits, hyperparameters, how they were chosen)?
        \answerTODO{}
      \item Did you report error bars (e.g., with respect to the
        random seed after running experiments multiple times)?
        \answerTODO{}
      \item Did you include the total amount of compute and the type
        of resources used (e.g., type of GPUs, internal cluster, or
        cloud provider)?
        \answerTODO{}
    \end{enumerate}

  \item If you are using existing assets (e.g., code, data, models)
    or curating/releasing new assets...
    \begin{enumerate}
      \item If your work uses existing assets, did you cite the creators?
        \answerTODO{}
      \item Did you mention the license of the assets?
        \answerTODO{}
      \item Did you include any new assets either in the
        supplemental material or as a URL?
        \answerTODO{}
      \item Did you discuss whether and how consent was obtained
        from people whose data you're using/curating?
        \answerTODO{}
      \item Did you discuss whether the data you are using/curating
        contains personally identifiable information or offensive content?
        \answerTODO{}
    \end{enumerate}

  \item If you used crowdsourcing or conducted research with human subjects...
    \begin{enumerate}
      \item Did you include the full text of instructions given to
        participants and screenshots, if applicable?
        \answerTODO{}
      \item Did you describe any potential participant risks, with
        links to Institutional Review Board (IRB) approvals, if applicable?
        \answerTODO{}
      \item Did you include the estimated hourly wage paid to
        participants and the total amount spent on participant compensation?
        \answerTODO{}
    \end{enumerate}

\end{enumerate}

%%%%%%%%%%%%%%%%%%%%%%%%%%%%%%%%%%%%%%%%%%%%%%%%%%%%%%%%%%%%

\appendix

\section{Appendix}

Optionally include extra information (complete proofs, additional
experiments and plots) in the appendix.
This section will often be part of the supplemental material.

\end{document}
